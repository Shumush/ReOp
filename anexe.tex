\documentclass[a4paper, 11pt]{article}

\usepackage[utf8]{inputenc}
\usepackage[T1]{fontenc}
\usepackage[francais]{babel}
\usepackage{amsfonts}
\usepackage{sectsty}
\usepackage{amsmath, amssymb}

% custom package to speed up writing math objects
\usepackage{speedmath}

\sectionfont{\fontsize{14}{15}\selectfont}
\subsectionfont{\fontsize{12}{13}\selectfont}

\begin{document}

\title{Introduction au calcul matriciel et à l'algèbre linéaire}
\author{Arthur Blanleuil}

\maketitle

\section{Shoubidou !}

La recherche opérationnelle s'appuie énormément sur les outils mathématiques, notemment l'algèbre linéaire
de ce fait, un minimum de connaissance dans ce domaine s'impose. Le but de cet article est d'apprendre ce
$"minimum"$ à toute personne qui voudrait apprendre les méthodes de résolutions de programmes linéaires. \\ \\

Il y aura beaucoup d'allusions à l'informatique, par exemple, nous associeront les matrices à des tableaux à
deux dimensions.

\section{Vecteurs et Matrices}

\subsection{Les vecteurs}

Un \underline{vecteur} représente un assortiment de valeurs, chacunes appartenant à des ensembles.
Par exemple, les coordonnées d'un point dans l'espace peuvent être représentées par un vecteur contenant
3 réels.

\[
  \vec{p} = \cvthree{x}{y}{z}
\]

Ses trois composantes sont dans l'ensemble des réels (on note $x, y, z \in \R$), par conséquent le vecteur $\vec{p}$
se trouve dans l'ensemble $\R \times \R \times \R = \R^3$. On note $\vec{p} \in \R^3$.

En C, on peut voir un vecteur comme une structure. Un vecteur comme $\vec{p}$ peut se représenter sous la forme d'une matrice
contenant 3 \texttt{double}s

Généralement, en mathématiques, les valeurs d'un vecteur viennent du même ensemble, on a rarement un vecteur qui contient un
réel et un complexe.

Les valeurs d'un vecteur $\vec{v} \in \K^n$ donné sont appelés des \underline{scalaires} dans l'ensemble $\K$

\newpage

\subsection{Les matrices}

Les matrices sont des objets mathématiques qui peuvent s'apparenter à des tableaux à deux dimensions.
Une matrice a un nombre de lignes, et un nombre de colonnes caractérisants ses dimensions.

L'ensemble des matrices avec $m$ lignes, et $n$ colonnes, dont les valeurs sont dans l'ensemble $\K$ se note ainsi: $\Mat{m}{n}{\K}$

Une telle matrice $A$ appartenant donc à cet ensemble se note $A \in \Mat{m}{n}{\K}$. \\

Soit $A \in \Mat{2}{2}{\R}$, $A = \begin{pmatrix} a & b \\ c & d \end{pmatrix}$.
Alors $a$, $b$, $c$, et $d$ sont des réels, et sont appelés les coefficients de $A$

\subsubsection{Notations sur les matrices}

On appelle $A_i$ la $i$-ème ligne de la matrice $A$.
On appelle $A^{j}$ la $j$-ième colonne de la matrice $A$. \\

Par composition, on appelle $A_{i}^{j}$ la valeur de la matrice qui se trouve
à la ligne $i$ et à la colonne $j$ (en C, on écrit \texttt{A[i][j]}). \\

On peut non pas utiliser des indices seuls, mais des ensembles d'indices.

Par exemple, si j'appelle $I$ un ensemble d'indices entre $1$ et $m$,
la matrice $A_{I}$ est la matrice qui contient toutes les lignes de $A$
dont les indices se trouvent dans $I$.

Par conséquent, une telle matrice $A_{I}$ contient $\abs{I}$ lignes
(la notation $\abs{I}$ représente le cardinal de $I$, qui est le nombre d'élements
dans $I$) \\

Soit $A \in \Mat{3}{3}{\R} \; ; \; A = \begin{pmatrix} a&b&c\\d&e&f\\g&h&i\end{pmatrix}$
et $I = \{1, 3\}$, la matrice $A_{I}$ est alors égale à $\begin{pmatrix}a&b&c\\g&h&i\end{pmatrix}$ \\

On note de la même manière pour les colonnes.
Si on prend la matrice $A$ de l'exemple précédent, et $J = \{1, 2\}$ alors $A^{J} = \begin{pmatrix}a&b\\d&e\\g&h\end{pmatrix}$

\newpage
\subsubsection{Opérations sur les matrices}

L'addition s'effectue termes par termes, et les matrices DOIVENT avoir le même nombre de colonnes et de lignes.
Cette opération est commutative ($A + B = B + A$) et associative ($A + (B + C) = (A + B) + C$)

Soient $A, B, C \in \Mat{2}{2}{\R} \; ; \; A = \begin{pmatrix}a&b\\c&d\end{pmatrix} \; ; \; B =
\begin{pmatrix}a'&b'\\c'&d'\end{pmatrix} \; ; \; C = A + B$.
Alors $C = \begin{pmatrix}a+a'&b+b'\\c+c'&d+d'\end{pmatrix}$ \\ \\

La multiplication est non commutative ($A \times B \ne B \times A$), mais elle est associative.

L'opération $A \times B$ ne peut s'effectuer que si le nombre de colonnes de $A$ est égal au nombre de lignes de $B$

Si $A \in \Mat{m}{n}{\R}$ et $B \in \Mat{n}{p}{\R}$, alors l'opération $A \times B$ est possible, et la matrice résultante
de cette opération est une matrice avec $m$ lignes et $p$ colonnes. \\ \\

Soit $C$ la matrice résultante de $A \times B$. Alors $C \in \Mat{m}{p}{\R}$ \\
et $C_{i}^{j} = \sum_{k = 1}^{n} A_{i}^{k} B_{k}^{j}$. \\ \\

La valeur de la case au croisement de la ligne $i$ et la colonne $j$ est l'addition des coefficients de la ligne $A_i$
multipliés uns à uns par ceux de la colonne $B^{j}$

\end{document}
