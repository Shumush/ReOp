\documentclass[a4paper, 11pt]{article}

\usepackage[utf8]{inputenc}
\usepackage[T1]{fontenc}
\usepackage[francais]{babel}
\usepackage{amsfonts}
\usepackage{sectsty}
\usepackage{amsmath, amssymb}

% custom package to speed up writing math objects
\usepackage{speedmath}

\sectionfont{\fontsize{14}{15}\selectfont}
\subsectionfont{\fontsize{12}{13}\selectfont}

\begin{document}

\title{Recherche Operationelle\\Chapitre 2: Résolution de programmes linéaires}
\author{Arthur Blanleuil}

\maketitle

\section{Quelques Théorèmes}

\subsection{Théorème 1}

Soit $A \in \M_{m,n}(\R)$, $\vec{b}$ un vecteur colonne de dimension m, pour la suite du cours, on considère le système linéaire suivant:

\begin{equation}A\vec{x} = \vec{b}\end{equation}

\subsubsection{Définition 1}

Deux systèmes linéaires $A\vec{x} = \vec{b}$ et $A'\vec{x} = \vec{b'}$ sont dits équivalents si et seulement si ils ont les mêmes solutions.

\subsubsection{Théorème 1}

Etant donné $(1)$, on a trois cas possibles concernant le rang de $A$:

\begin{itemize}
  \item $rang(A) = m$ : on dit que le système est de \textbf{plein rang}
    dans ce cas, $m \le n$ et l'ensemble des solutions $S = \{ \vec{x} \; ; \; A\vec{x} = \vec{b} \}$ est non nul
    et $|S| = 1 \Leftrightarrow m = n$ (si la solution est unique, alors $A$ est une matrice carré)
  \item $rang(A) < m$ et le système n'a pas de solutions alors $\exists \vec{y} \; ; \; \vec{y}A = \vec{0}$ et $\vec{y} . \vec{b} \ne 0$
  \item $rang(A) < m$ et le système a des solutions alors $\exists \vec{y} \ne \vec{0} \; ; \; \vec{y}A = \vec{0}$ et $\vec{y} . \vec{b} \ne 0$
    ET $\exists I \in \{ 1 .. m \} \; ; \; A_I\vec{x} = \vec{b_I}$ et $A$ est de plein rang.
\end{itemize}


\subsubsection{Remarque}
$I$ n'est pas unique en général.

Si les systèmes $A\vec{x} = \vec{b}$ et $A'\vec{x} = \vec{b'}$ sont équivalents, alors il éxiste une matrice $B \in \M_{m,m}(\R)$ inversible
telle que $A' = BA$ et $\vec{b'} = B\vec{b}$

\subsubsection{Définition 2}

La matrice $(A,\vec{b}) \in \M_{m,n+1}(\R)$ est appelée la \underline{matrice augmentée} du système linéaire $(1)$

\subsection{Théorème 2}

Soit $B \in \M_{m,m}(\R)$ une matrice non singulère (c'est à dire $rang(B) = m$)
le système linéaire $BA\vec{x} = B\vec{b}$ est équivalent à (1)

Dit autrement, toute transformation linéaire de la matrice et du vecteur contrainte d'un système linéaire
donne un système équivalent.

\subsubsection{Remarque}


Les transformations équivalentes d'un système linéaire à un autre peuvent s'exprimer sous la forme
de matrices "presque comme" des matrice identités. On distingue 2 cas très utiles plus tard quand
nous aborderons la méthode du simplex, pour faire apparaitre une certaine "forme" de système.


\begin{enumerate}
  \item $B$ est une matrice de cette forme:
    \[
      \begin{pmatrix}
        1      & 0      & \cdots    & \cdots & 0      \\
        0      & \ddots & \ddots    &        & \vdots \\
        \vdots & \ddots & B_{r}^{r} & \ddots & \vdots \\
        \vdots &        & \ddots    & \ddots & 0      \\
        0      & \cdots & \cdots    & 0      & 1 
      \end{pmatrix}
    \] \\
    c'est à dire une matrice consistuée d'une diagonale de 1, sauf à $B_{r}^{r}$ où $r = rang(A)$
    à ce moment, $B$ est de plein rang si et seulement si $B_{r}^{r} \ne 0$ (En effet, le vecteur nul ne peut faire partie d'une base).

    Ce genre de matrice ne fait que modifier une ligne du système d'équations en la multipliant par un réel.
    Cette transformation est utile pour faire apparaitre un 1 dans les coefficients d'une ligne.

  \item $B$ est une matrice identitée avec la valeur $B_{k}^{r} = \alpha \ne 0$. Cette forme permet de remplacer la $r$-ième équation
    (qu'on appellera $L_{r}$) par cette valeur : $L'_{r} = L_{r} + \alpha L_{k}$. On peut ainsi faire apparaitre des 0 à des endroits de
    $L_{r}$. \\
    Attention, ici $r$ n'a rien a voir avec le rang de $B$.
\end{enumerate}

\subsection{Définition 3}

Résoudre le système linéaire $A\vec{x} = \vec{b}$ c'est:
\begin{itemize}
  \item Démontrer qu'il n'a pas de solution
  \item Trouver un système équivalent tel que $\exists \; J \in \{ 1 ... n \} \;$tel que$\; A'^{J}$ est à une permutation de colonne près d'être l'identité.
\end{itemize}

\end{document}
