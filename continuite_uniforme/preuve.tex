\documentclass[a4paper, 14pt]{article}

\usepackage[utf8]{inputenc}
\usepackage[T1]{fontenc}
\usepackage{fontspec}
\usepackage[francais]{babel}
\usepackage{amsfonts}
\usepackage{sectsty}
\usepackage{amsmath, amssymb}

\usepackage{speedmath}

%\sectionfont{\fontsize{14}{15}\selectfont}
%\subsectionfont{\fontsize{12}{13}\selectfont}

\begin{document}

\title{Continuité uniforme de $x \to x^2$}
\author{Arthur Blanleuil}

\maketitle

\section{Définitions}

On rappelle qu'une fonction $f : \set{A} \to \set{B}$ est uniformément continue
dans un intervalle $\set{I} \subseteq \set{A}$ si et seulement si:

\[
  \forall \epsilon > 0, \exists \delta > 0 \;\;|\;\; \forall (x, y) \in
  \set{I}^2,\; \abs{x - y} < \delta \implies \abs{f(x) - f(y)} < \epsilon
\]

Autrement dit, pour tout $\epsilon$ assez petit (mais non nul), il existe une distance non
nulle $\delta$ telle que pour tout $x$ et $y$ dans $\set{I}$, si la distance de
$x$ à $y$ est plus petite que $\delta$ alors la distance entre $f(x)$ et $f(y)$
est plus petite que $\epsilon$.

\section{Dans $\R$}

Supposons $f : x \to x^2$ uniformément continue dans $\R$. Prenons
$\epsilon = 1$. Si $f$ est uniformément continue, alors il existe un $\delta$
tel que

\[
  \forall (x, y) \in \R^2, \;\; \abs{x - y} < \delta
  \implies \abs{x^2 - y^2} < 1
\]

Prenons 
\[
  x = \frac{1}{\delta}\;\;\;\;\;\; y = \frac{1}{\delta} + \frac{\delta}{2}
\]

Alors

\[
  \abs{x - y} = \frac{\delta}{2} < \delta
\]

Vérifions que ces valeurs vérifient la continuité uniforme: $\abs{x^2 - y^2} < 1$
\begin{align*}
  \abs{x^2 - y^2} &= \left\lvert \left(\frac{1}{\delta}\right)^2 - \left(\frac{1}{\delta} +
  \frac{\delta}{2}\right)^2 \right\lvert \\
  &= \left\lvert \frac{1}{\delta^2} - \frac{1}{\delta^2} -
    2\frac{1}{\delta}\frac{\delta}{2} - \frac{\delta^2}{4} \right\lvert\\
  &= \left\lvert - 1 - \frac{\delta^2}{4} \right\lvert\\
  &= \left\lvert - \left(1 + \frac{\delta^2}{4} \right) \right\lvert \\
  \abs{x^2 - y^2} &= \boxed{1 + \frac{\delta^2}{4}}
\end{align*}

Nous avons trouvé un $x$ et un $y$ qui contredisent la définition de la
continuité uniforme pour $\epsilon = 1$. La fonction $f : x \to x^2$ n'est donc
\textbf{pas uniformément continue sur $\R$}.

\section{Dans un intervalle fermé borné $\left[a;\;b\right]$}

Intéressons-nous à $f : x \to x^2$ sur l'intervalle $\left[a;\;b\right]$.
Prenons la borne de notre intervalle la plus
éloignée de $0$ : $A = max(\abs{a}, \abs{b})$.

Par l'inégalité triangulaire de la valeur absolue [1], on a:

\begin{align}
  &\forall x \in [a;\;b], \; \abs{x} \le A\\
  &\forall (x, y) \in [a;\;b]^2, \; \abs{x + y} \le \abs{x} + \abs{y} \le A+A = 2A
\end{align}

Si pour tout $\epsilon > 0$, on prend
\[
  \delta = \frac{\epsilon}{2A}
\]

Alors pour tout $(x, y) \in [a;\;b]^2$ on a:
\begin{align*}
  \abs{x^2 - y^2} &= \abs{(x - y)(x + y)}&\\
                  &= \abs{x-y}\abs{x+y}&[2]\\
                  &\le \abs{x-y}2A &[3]\\
                  &< \frac{\epsilon}{2A}2A &[4]\\
                  &< \epsilon
\end{align*}

\bigskip

\fbox{
  \parbox{\textwidth}{
On a bien vérifié que la fonction
$f:x\to x^2$ est uniformément continue sur un intervalle fermé et borné
quelconque $[a;\;b]$.
}
}

\bigskip
\begin{itemize}
  \item[1] la valeur absolue est une distance dans \R, et toute distance vérifie
    l'inégalité triangulaire.
  \item[2] $\abs{AB} = \abs{A}\abs{B}$ (Tu peux le prouver par énumération
    sur les combinaisons de signes de $A$ et de $B$).
  \item[3] Une valeur absolue est forcément positive, alors la remplacer par
    une quantité plus petite mais toujours positive ($A$, d'après l'équation 2)
    modifie le signe de l'égalité.
  \item[4] On se permet de passer de $\le$ à $<$ parce que d'après la
    définition de la continuité uniforme, on ne s'occupe que des couples $(x,y)$ qui
    vérifient $\abs{x - y} < \delta$
\end{itemize}
\end{document}
